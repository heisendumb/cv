%%%%%%%%%%%%%%%%%%%%%%%%%%%%%%%%%%%%%%%%%
% Friggeri Resume/CV
% XeLaTeX Template
% Version 1.0 (5/5/13)
%
% This template has been downloaded from:
% http://www.LaTeXTemplates.com
%
% Original author:
% Adrien Friggeri (adrien@friggeri.net)
% https://github.com/afriggeri/CV
%
% License:
% CC BY-NC-SA 3.0 (http://creativecommons.org/licenses/by-nc-sa/3.0/)
%
% Important notes:
% This template needs to be compiled with XeLaTeX and the bibliography, if used,
% needs to be compiled with biber rather than bibtex.
%
%%%%%%%%%%%%%%%%%%%%%%%%%%%%%%%%%%%%%%%%%

\documentclass[]{friggeri-cv} % Add 'print' as an option into the square bracket to remove colors from this template for printing

\begin{document}

\header{guilherme}{de albuquerque}{analista de infraestrutura} % Your name and current job title/field


%----------------------------------------------------------------------------------------
%	SIDEBAR SECTION
%----------------------------------------------------------------------------------------

\begin{aside} % In the aside, each new line forces a line break
\section{contato}
+(55)(48)996494146
~
\href{mailto:guiealbuquerque@gmail.com}{guiealbuquerque@gmail.com}
\href{https://github.com/heisendumb}{GitHub}
\href{https://www.linkedin.com/in/heisenbugger/}{LinkedIn}
\section{idiomas}
Inglês: avançado (leitura, escrita); intermediário (fala)
\section{linguagens de desenvolvimento de software}
C++, Python, Shell
\section{protocolos}
SIP, RTC, HTTP, SMTP, RTP, SNMP
\section{nuvem}
OpenStack, Azure, AWS
\section{orquestração de contêineres}
OpenShift/Kubernetes
\section{banco de dados}
PostgreSQL, MySQL
\section{automação}
Ansible, Terraform, Groovy
\section{ambiente de desenvolvimento}
Vagrant, Docker Compose
\section{monitoramento}
ELK, Grafana/Prometheus, Sysdig, Datadog
\end{aside}

%----------------------------------------------------------------------------------------
%	EDUCATION SECTION
%----------------------------------------------------------------------------------------

\section{formação}

\begin{entrylist}
%------------------------------------------------
%------------------------------------------------
\entry
{2012--2019}
{Engenharia de Telecomunicações}
{IFSC}
{
    \begin{itemize}
        \item \href{https://wiki.sj.ifsc.edu.br/index.php/Matriz_Curricular_da_Engenharia_de_Telecomunica%C3%A7%C3%B5es}{Matriz curricular}
    \end{itemize}
}
%------------------------------------------------
\end{entrylist}

%----------------------------------------------------------------------------------------
%	WORK EXPERIENCE SECTION
%----------------------------------------------------------------------------------------

\section{experiências}

\begin{entrylist}
%------------------------------------------------
\entry
{2018--agora}
{Grupo Nexxera}
{Florianópolis, Santa Catarina}
{\emph{Analista de Infraestrutura} \\

    Como membro da área de infraestrutura sou responsável pela infraestrutura em nuvem,
    processos de automação e o monitoramento. Além disso, atuo como evangelista da cultura DevOps e juntamente com as equipes de desenvolvimento implemento os processos de integração contínua e entrega contínua. Para desenvolver tais ideias e melhorar os processos internos de desenvolvimento, houve vários problemas complicados a serem resolvidos como:

\begin{itemize}
    \item 
\end{itemize}

Principais realizações:\\

\begin{itemize}
    \item 
\end{itemize}

}
\end{entrylist}

\begin{entrylist}
%------------------------------------------------
\entry
{2017--2018}
{Way2 Tecnologia}
{Florianópolis, Santa Catarina}
{\emph{Engenheiro de Software} \\

    Como membro e líder na área de infraestrutura fui responsável pela infraestrutura em nuvem, implantação on-premise,
    processos de automação e o monitoramento. Além disso, atuo como evangelista da cultura DevOps e juntamente com as equipes de desenvolvimento e
    de qualidade desenvolvo e implemento os processos de integração contínua e entrega contínua. Para desenvolver tais ideias e melhorar os processos
    internos de desenvolvimento, houve vários problemas complicados a serem resolvidos como:

\begin{itemize}
    \item Especificação de uma DSL para desenvolver os pipelines de entrega contínua
    \item Definir soluções para implantação, ambientes de desenvolvimento e produção que independem da tecnologia utilizada no desenvolvimento dos produtos
    \item Problemas de performance e de segurança com Oracle na Azure
\end{itemize}

Principais realizações:\\

\begin{itemize}
    \item Implementação de monitoramento em tempo real para ambientes de produção utilizando a stack ELK (Elasticsearch, Logstash e Kibana)
    \item Participação ativa no desenvolvimento de web crawlers utilizados na etapa de qualidade do pipeline de entrega contínua
    \item Automação do ambiente de desenvolvimento utilizando docker compose
    \item Orquestração de cluster na Azure utilizando Kubernetes e Docker
    \item Automação da implantação em clientes on-premise utilizando Octopus
    \item Implementação de pipelines para entrega contínua dos projetos
\end{itemize}

}
\end{entrylist}

\begin{entrylist}
%------------------------------------------------
\entry
{2015--2017}
{Disruptiva - Franchise Intelligence}
{Florianópolis, Santa Catarina}
{\emph{Engenheiro de Software} \\

    Como membro responsável da área de DevOps, fui designado a implantar a estrutura em nuvem
    (web, android e IOS), automação dos processos e monitoramento. \\
    Além disso, atuei como evangelista da cultura DevOps e juntamente com os desenvolvedores implementamos 
    os processos de integração contínua e entrega contínua. 
    Houve muitos desafios como:\\

\begin{itemize}
    \item Implantação de toda a infraestrutura de múltiplos projetos na Azure
    \item Definição da topologia de rede e medidas de segurança
    \item Criação de rotinas de backups para as VMs com diferentes sistemas operacionais na Azure
    \item Versionamento da implantação da infraestrutura para criação de múltiplos ambientes simultâneos
\end{itemize}

Principais realizações:

\begin{itemize}
    \item Automação do ambiente de desenvolvimento utilizando docker compose
    \item Painel de monitoramento totalmente automatizado utilizando logentries e datadog
    \item Automatizado o processo provisionamento de criação de VMs na Azure
    \item Versionamento da implantação de toda infraestrutura e maneira simples de replicação da mesma (incluindo rede e segurança)
    \item Participação ativa no desenvolvimento de ferramentas para automatizar a infraestrutura na Azure
    \item Implementação de pipelines para entrega contínua dos produtos
    \item Replicação de banco de dados utilizando MySQL e PostgreSQL
\end{itemize}
}
\end{entrylist}

\begin{entrylist}
%------------------------------------------------
\entry
{2014--2015}
{IFSC | Tractebel S.A}
{São José, Santa Catarina}
{\emph{Bolsista} \\
    
    Desenvolvimento de um protótipo analisador de espectro para medir rendimento de motores elétricos trifásico.

Principais realizações:\\

\begin{itemize}
    \item Implementação de uma aplicação para análise espectral utilizando C++ e QtCreator em Linux
    \item Implementação de uma aplicação para captura de dados oriundos de sensores através de um conversor AD
    \item Desenvolvimento de uma estrutura relacional em SQL para armazenamento dos dados obtidos dos sensores utilizando a biblioteca SQLite em Python.
\end{itemize}
}
\end{entrylist}
\pagebreak

%----------------------------------------------------------------------------------------
%	PRESENTATIONS/EVENTS SECTION
%----------------------------------------------------------------------------------------

\section{eventos}

%------------------------------------------------
\begin{entrylist}
\entry
{2014}
{Competição Intel de Sistemas Embarcados}
{\href{http://sbesc.lisha.ufsc.br/sbesc2014/Intel+Embedded+Systems+Competition}{SBESC2014}}
{
    
    Participação na competição de sistemas embarcados promovida pela Intel 
    e representado o IFSC/SJ que ocorreu no SBESC2014.\\

    Nessa competição foi submetido o projeto de Identificação Automática de Símbolos da linguagem de sinais por vídeo,
    desenvolvido em C/C++ utilizando OpenCV e QtCreator em Linux.

}
%------------------------------------------------
\end{entrylist}

\pagebreak

%----------------------------------------------------------------------------------------
%	INTERESTS SECTION
%----------------------------------------------------------------------------------------

\section{interests}

\textbf{profissional:} sistemas distribuídos, computação em nuvem, test driven development, arquitetura de software, redes de computadores, linux, open source, inteligência artificial.

\textbf{livros:} Hackers: Heroes Of The Computer Revolution, REST in Practice, Site Reliability Engineering, Fundamentos de Sistemas Operacionais.

\end{document}
